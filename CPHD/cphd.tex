%%%%%%%%%%%%%%%%%%%%%%%%%%%%%%%%%%%%%%%%%
% KOMA-Script Presentation
% LaTeX Template
% Version 1.0 (3/3/13)
%
% This template has been downloaded from:
% http://www.LaTeXTemplates.com
%
% Original Authors:
% Marius Hofert (marius.hofert@math.ethz.ch)
% Markus Kohm (komascript@gmx.info)
% Described in the PracTeX Journal, 2010, No. 2
%
% License:
% CC BY-NC-SA 3.0 (http://creativecommons.org/licenses/by-nc-sa/3.0/)
%
%%%%%%%%%%%%%%%%%%%%%%%%%%%%%%%%%%%%%%%%%

%----------------------------------------------------------------------------------------
%	PACKAGES AND OTHER DOCUMENT CONFIGURATIONS
%----------------------------------------------------------------------------------------

\documentclass[
paper=128mm:96mm, % The same paper size as used in the beamer class
fontsize=11pt, % Font size
pagesize, % Write page size to dvi or pdf
parskip=half-, % Paragraphs separated by half a line
]{scrartcl} % KOMA script (article)

\linespread{1.12} % Increase line spacing for readability

%------------------------------------------------
% Colors
\usepackage{xcolor}	 % Required for custom colors
\usepackage{amsmath}
% Define a few colors for making text stand out within the presentation
\definecolor{mygreen}{RGB}{44,85,17}
\definecolor{myblue}{RGB}{34,31,217}
\definecolor{mybrown}{RGB}{194,164,113}
\definecolor{myred}{RGB}{255,66,56}
% Use these colors within the presentation by enclosing text in the commands below
\newcommand*{\mygreen}[1]{\textcolor{mygreen}{#1}}
\newcommand*{\myblue}[1]{\textcolor{myblue}{#1}}
\newcommand*{\mybrown}[1]{\textcolor{mybrown}{#1}}
\newcommand*{\myred}[1]{\textcolor{myred}{#1}}
%------------------------------------------------

%------------------------------------------------
% Margins
\usepackage[ % Page margins settings
includeheadfoot,
top=3.5mm,
bottom=3.5mm,
left=5.5mm,
right=5.5mm,
headsep=6.5mm,
footskip=8.5mm
]{geometry}
%------------------------------------------------

%------------------------------------------------
% Fonts
\usepackage[T1]{fontenc}	 % For correct hyphenation and T1 encoding
\usepackage{lmodern} % Default font: latin modern font
%\usepackage{fourier} % Alternative font: utopia
%\usepackage{charter} % Alternative font: low-resolution roman font
\renewcommand{\familydefault}{\sfdefault} % Sans serif - this may need to be commented to see the alternative fonts
%------------------------------------------------

%------------------------------------------------
% Various required packages
\usepackage{amsthm} % Required for theorem environments
\usepackage{bm} % Required for bold math symbols (used in the footer of the slides)
\usepackage{graphicx} % Required for including images in figures
\usepackage{tikz} % Required for colored boxes
\usepackage{booktabs} % Required for horizontal rules in tables
\usepackage{multicol} % Required for creating multiple columns in slides
\usepackage{lastpage} % For printing the total number of pages at the bottom of each slide
\usepackage[english]{babel} % Document language - required for customizing section titles
\usepackage{microtype} % Better typography
\usepackage{tocstyle} % Required for customizing the table of contents
%------------------------------------------------

%------------------------------------------------
% Slide layout configuration
\usepackage{scrpage2} % Required for customization of the header and footer
\pagestyle{scrheadings} % Activates the pagestyle from scrpage2 for custom headers and footers
\clearscrheadfoot % Remove the default header and footer
\setkomafont{pageheadfoot}{\normalfont\color{black}\sffamily} % Font settings for the header and footer

% Sets vertical centering of slide contents with increased space between paragraphs/lists
\makeatletter
\renewcommand*{\@textbottom}{\vskip \z@ \@plus 1fil}
\newcommand*{\@texttop}{\vskip \z@ \@plus .5fil}
\addtolength{\parskip}{\z@\@plus .25fil}
\makeatother

% Remove page numbers and the dots leading to them from the outline slide
\makeatletter
\newtocstyle[noonewithdot]{nodotnopagenumber}{\settocfeature{pagenumberbox}{\@gobble}}
\makeatother
\usetocstyle{nodotnopagenumber}

\AtBeginDocument{\renewcaptionname{english}{\contentsname}{\Large Outline}} % Change the name of the table of contents
%------------------------------------------------

%------------------------------------------------
% Header configuration - if you don't want a header remove this block
\ihead{
\hspace{-2mm}
\begin{tikzpicture}[remember picture,overlay]
\node [xshift=\paperwidth/2,yshift=-\headheight] (mybar) at (current page.north west)[rectangle,fill,inner sep=0pt,minimum width=\paperwidth,minimum height=2\headheight,top color=mygreen!64,bottom color=mygreen]{}; % Colored bar
\node[below of=mybar,yshift=3.3mm,rectangle,shade,inner sep=0pt,minimum width=128mm,minimum height =1.5mm,top color=black!50,bottom color=white]{}; % Shadow under the colored bar
shadow
\end{tikzpicture}
\color{white}\runninghead} % Header text defined by the \runninghead command below and colored white for contrast
%------------------------------------------------

%------------------------------------------------
% Footer configuration
\newlength{\footheight}
\setlength{\footheight}{8mm} % Height of the footer
\addtokomafont{pagefoot}{\footnotesize} % Small font size for the footnote

\ifoot{% Left side
\hspace{-2mm}
\begin{tikzpicture}[remember picture,overlay]
\node [xshift=\paperwidth/2,yshift=\footheight] at (current page.south west)[rectangle,fill,inner sep=0pt,minimum width=\paperwidth,minimum height=3pt,top color=mygreen,bottom color=mygreen]{}; % Green bar
\end{tikzpicture}
\myauthor\ \raisebox{0.2mm}{$\bm{\vert}$}\ \myuni % Left side text
}

\ofoot[\pagemark/\pageref{LastPage}\hspace{-2mm}]{\pagemark/\pageref{LastPage}\hspace{-2mm}} % Right side
%------------------------------------------------

%------------------------------------------------
% Section spacing - deeper section titles are given less space due to lesser importance
\usepackage{titlesec} % Required for customizing section spacing
\titlespacing{\section}{0mm}{0mm}{0mm} % Lengths are: left, before, after
\titlespacing{\subsection}{0mm}{0mm}{-1mm} % Lengths are: left, before, after
\titlespacing{\subsubsection}{0mm}{0mm}{-2mm} % Lengths are: left, before, after
\setcounter{secnumdepth}{0} % How deep sections are numbered, set to no numbering by default - change to 1 for numbering sections, 2 for numbering sections and subsections, etc
%------------------------------------------------

%------------------------------------------------
% Theorem style
\newtheoremstyle{mythmstyle} % Defines a new theorem style used in this template
{0.5em} % Space above
{0.5em} % Space below
{} % Body font
{} % Indent amount
{\sffamily\bfseries} % Head font
{} % Punctuation after head
{\newline} % Space after head
{\thmname{#1}\ \thmnote{(#3)}} % Head spec
	
\theoremstyle{mythmstyle} % Change the default style of the theorem to the one defined above
\newtheorem{theorem}{Theorem}[section] % Label for theorems
\newtheorem{remark}[theorem]{Remark} % Label for remarks
\newtheorem{algorithm}[theorem]{Algorithm} % Label for algorithms
\makeatletter % Correct qed adjustment
%------------------------------------------------

%------------------------------------------------
% The code for the box which can be used to highlight an element of a slide (such as a theorem)
\newcommand*{\mybox}[2]{ % The box takes two arguments: width and content
\par\noindent
\begin{tikzpicture}[mynodestyle/.style={rectangle,draw=mygreen,thick,inner sep=2mm,text justified,top color=white,bottom color=white,above}]\node[mynodestyle,at={(0.5*#1+2mm+0.4pt,0)}]{ % Box formatting
\begin{minipage}[t]{#1}
#2
\end{minipage}
};
\end{tikzpicture}
\par\vspace{-1.3em}}
%------------------------------------------------

%----------------------------------------------------------------------------------------
%	PRESENTATION INFORMATION
%----------------------------------------------------------------------------------------

\newcommand*{\mytitle}{ parallel CPHD filter} % Title
\newcommand*{\runninghead}{parallel CPHD filter} % Running head displayed on almost all slides
\newcommand*{\myauthor}{wang junjie} % Presenters name(s)
\newcommand*{\mydate}{\today} % Presentation date
\newcommand*{\myuni}{HIT --- School of Software  } % University or department

%----------------------------------------------------------------------------------------

\begin{document}

%----------------------------------------------------------------------------------------
%	TITLE SLIDE
%----------------------------------------------------------------------------------------

% Title slide - you may have to tweak a few of the numbers if you wish to make changes to the layout
\thispagestyle{empty} % No slide header and footer
\begin{tikzpicture}[remember picture,overlay] % Background box
\node [xshift=\paperwidth/2,yshift=\paperheight/2] at (current page.south west)[rectangle,fill,inner sep=0pt,minimum width=\paperwidth,minimum height=\paperheight/3,top color=mygreen,bottom color=mygreen]{}; % Change the height of the box, its colors and position on the page here
\end{tikzpicture}
% Text within the box
\begin{flushright}
\vspace{0.6cm}
\color{white}\sffamily
{\bfseries\Large\mytitle\par} % Title
\vspace{0.5cm}
\normalsize
\myauthor\par % Author name
\mydate\par % Date
\vfill
\end{flushright}

\clearpage

%----------------------------------------------------------------------------------------
%	TABLE OF CONTENTS
%----------------------------------------------------------------------------------------

\thispagestyle{empty} % No slide header and footer

\small\tableofcontents % Change the font size and print the table of contents - it may be useful to shrink the font size further if the presentation is full of sections
% To exclude sections/subsections from the table of contents, put an asterisk after \(sub)section like so: \section*{Section Name}

\clearpage

%----------------------------------------------------------------------------------------
%	PRESENTATION SLIDES
%----------------------------------------------------------------------------------------

\section{CPHD filter}
Analytic Implementations of the Cardinalized Probability Hypothesis Density Filter
\texttt{Ba-Tuong Vo, Ba-Ngu Vo, and Antonio Cantoni} 2007
\clearpage

%------------------------------------------------

\subsection{introduction}

 Due to the inherent conbinatorial nature of multitarget densities and the multiple integrations on the multitarget state and observation space,the multitarget Bayes recursion is intractable in most ptactical applications. To alleviate this intractability the PHD recursion was developd as a first moment approximation to the multiarget bayes recursion.\\
THe PHD recursion propagates cardinality information with only a single parameter,and thus,it effectively approximates the
cardinality distribution by a Poisson distribution.

\textit{When the number of targets present is high,the PHD filter estimates the cardinality with a correspondingly high variance.
}

%\begin{multicols}{2} % Divide text into multiple columns
%\mygreen{Sed diam enim, sagittis nec} condimentum sit amet, ullamcorper sit amet libero. \mybrown{Aliquam vel dui orci}, a porta odio. %\myred{Nullam id suscipit} ipsum. \myblue{Aenean lobortis} commodo sem, ut commodo leo gravida vitae. Pellentesque vehicula ante iaculis arcu %pretium rutrum eget sit amet purus. Integer ornare nulla quis neque ultrices lobortis. Vestibulum ultrices tincidunt libero, quis commodo %erat ullamcorper id.
%\end{multicols}

\clearpage
%----------------------------------------------------
\subsection{RFSs}
suppose at time k there are N(k) targets with states $x_{k,1},....x_{k,N(k)}$ each taing values in a state space $ \chi\subseteq R$
Suppose also at time k that M(k) measurements $z_{k,1}.....z_{k,M(k)}$ are received each taking values in an observation space.

\begin{equation}
   X_k = \left\{ x_{k,1},...,x_{k,N(k)} \right\}  \in  f(\chi)
\end{equation}

\begin{equation}
   Z_k = \left\{ z_{k,1},...,z_{k,M(k)} \right\}  \in  f(z)
\end{equation}


 Intuitively, an RFS is simply a finite-set-valued random variable which can be completely
 characterized by a discrete probability distribution and a family of joint probability densities.



\clearpage
%------------------------------------------------
\subsection{RFSs}
we consider multitarget dynamics modeled by
\begin{equation}
   X_k = \left[\bigcup_{\zeta \in X_{k-1}}  S_{k|k-1}(\zeta)   \right] \bigcup { \Gamma_{k}}
\end{equation}

 Similarly,the multitarget sensor observations are modeled by
 \begin{equation}
   Z_k = \left[\bigcup_{x \in X_{k}}   \Theta_{k}(x)   \right] \bigcup { K_{k}}
\end{equation}
\clearpage
%------------------------------------------------
\subsection{RFSs}
\textit{multitarget transition density} $f_{k|k-1}(\bullet|\bullet)$ descirbes the time evolution of the
multitarget state and encapsulates the underlying models of target motions,birth and deaths.

Similarly, the \textit{likelihood} $g_k(\bullet|\bullet)$ describes the multitarget sensor measurement and encapsulates the underlying
models of detections,false alarms,and target generated measurements.





\clearpage
%------------------------------------------------
\subsection{RFSs}
The \textit{multitarget Bayes recursion} propagates the multitarget posterior density $\pi_k(\bullet|Z_{1:k})$
\begin{equation}
   \pi_{k|k-1}(X_{k} | Z_{1:k-1}) = \int f_{k|k-1}(X_k|X) \pi_{k-1}(X|Z_{1:k-1})\mu_s(dX)
\end{equation}

\begin{equation}
   \pi_{k}(X_{k} | Z_{1:k}) = \frac{g_k(Z_k|X_k)\pi_{k|k-1}(X_k|Z_{1:k-1})}{\int g_k(Z_k|X_k)\pi_{k|k-1}(X_k|Z_{1:k-1})\mu_s(dX)}
\end{equation}

\clearpage
%-------------------------------------------------
\subsection{solution to the CPHD recursion}
The CPHD recursion rests on the following assumptions:
\begin{itemize}
\item each target evolves and generates measurements independently of one another
\item the birth RFS and the surviving RFSs are independent of each other
\item the clutter RFS is an i.i.d cluster process and independent of measurement RFSs
\item the prior and predicted multitatget RFSs are i.i.d cluster processes.
\end{itemize}
\clearpage
%-------------------------------------------------
\subsection{solution to the CPHD recursion}
proposition 1: suppose at time k-1 that the posterior intensity $v_{k-1}$ and posterior cardiniality distribution $p_{k-1}$ are given.
\begin{equation}
   p_{k|k-1}(n) = \sum_{j=0}^{n} p_{\Gamma,k}(n-j)\Pi_{k|k-1}\left[ v_{k-1},p_{k-1} \right](j)
\end{equation}
\begin{equation}
   v_{k|k-1}(n) = \int p_{S,k}(\zeta) f_{k|k-1}(x|\zeta)v_{k-1}(\zeta)d\zeta + \gamma_{k}(x)
\end{equation}

where
$\Pi_{k|k-1}\left[ v,p \right](j) = \sum_{\iota = j}^{\infty} C_{j}^{\iota}  \frac{<p_{S,k},v>^{j} <1-p_{S,k},v>^{\iota-j}}{<1,v>^{\iota}} {p(\iota)}$
\clearpage
%-------------------------------------------------

\subsection{solution to the CPHD recursion}
proposition 2:  update
\begin{equation}
   p_{k}(n) = \frac{\Upsilon_{k}^{0}[v_{k|k-1},Z_k](n)p_{k|k-1}(n)}{<\Upsilon_{k}^{0}[v_{k|k-1},Z_k],p_{k|k-1}>}
\end{equation}
\begin{equation}
    v_{k}(x) = \frac{<\Upsilon_{k}^{1}[v_{k|k-1},Z_k],p_{k|k-1}>} {<\Upsilon_{k}^{0}[v_{k|k-1},Z_k],p_{k|k-1}>}\\
          \times[1-p_{D,k}(x)]v_{k|k-1}(x) \\
          + \sum_{z\in Z_k} \frac{<\Upsilon>}{}
\end{equation}

 \clearpage


\section{Closed-Form Solution to the CPHD Recursion}
Each target follows a linear Gaussian dynamical model
\begin{equation}
  f_{k|k-1}(x|\xi) = N(x;F_{k-1}\zeta,Q_{k-1})
\end{equation}
\begin{equation}
  g_{k}(z|x) = N(z;H_{k}x,R_k)
\end{equation}
where $N(;m,P)$ denotes a Gaussian density with mean m and covariance P,$F_{k-1}$ is the state transition matrix,


The intensity of the birth RFS is a Gaussian mixture of the form:
\begin{equation}
  \gamma_{k}(x) = \sum_{i=1}^{J_{\gamma,k}} \omega_{\gamma,k}^{(i)} N(x;m,P)
\end{equation}

where $\omega$ ,m,P are the weights,means,and covariances of the mixture birth intensity.
\clearpage
%-----------------------------------------------------
\subsection{Closed-Form Solution to the CPHD Recursion}
Suppose at time k-1 that the posterior intensity $v_{k-1}$ and posterior cardinality distribution $p_{k-1}$ are given,and that
$v_{k-1}$ is a Gaussian mixture of the form
\begin{equation}
   v_{k-1} = \sum_{i=1}^{j_{k-1}} \omega_{k-1}^{(i)}N(x;m_{k-1}^{(i),P_{k-1}^{(i)}})
\end{equation}
the CPHD prediction simplifies to
%\begin{equation}
\begin{gather}
p_{k|k-1}(n) = \sum_{j=0}^{n} p_{\Gamma,k}(n-j) \sum_{\iota=j}^{\infty}C_{j}^{\iota}p_{S,k}^{j}{(1-p_{s,k})}^{\iota-j}\\
v_{k|k-1}(x)=v_{S,k|k-1}(x) + \gamma_{k}(x)
\end{gather}

\clearpage
%\end{equation}
%-----------------------------------------------------
\subsection{Closed-Form Solution to the CPHD Recursion}
where
\begin{equation}
%v_{S,k|k-1}(x)=p_{S,k}\sum_{i=1}^{j_{k-1}}\omega_{k-1}^{(i)}N(x;m_{S,k|k-1}^{(i),P_{S,k|k-1}^{(i)}})\\
%m_{S,k|k-1}^{(i) = F_{k-1}m_{k-1}^{j}\\
%P_{S,k|k-1}^{(i) = Q_{k-1}+F_{k-1}P_{k-1}^{j}F_{k-1}^{T}
\end{equation}


\clearpage
%--------------------------------------------------
\section{code}
$x_k=[p_{x,k},p_{y,k},\overline{p}_{x,k},\overline{p}_{y,k}]$


$F_k=
\begin{bmatrix}
I_2 &  \Delta I_2 \\
0 & I_2
\end{bmatrix}
Q_k=\sigma_v^2\begin{bmatrix}
\frac{\Delta^4}{4} I_2 &  \frac{\Delta^3}{2} I_2 \\
\frac{\Delta^3}{2} I_z & {\Delta^2}I_2
\end{bmatrix}$

$p_{S,k}=0.99$

$H_k=[I_2 0_2]$
$R_k=\sigma_{\varepsilon}^{2}I_2$
\clearpage
%-------------------------------------------------
\subsection{Bullet Points and Numbered Lists}

\begin{itemize}
\item Lorem ipsum dolor sit amet, consectetur adipiscing elit
\item Aliquam blandit faucibus nisi, sit amet dapibus enim tempus eu
\end{itemize}

\begin{enumerate}
\item Nulla commodo, erat quis gravida posuere, elit lacus lobortis est, quis porttitor odio mauris at libero
\item Nam cursus est eget velit posuere pellentesque
\item Vestibulum faucibus velit a augue condimentum quis convallis nulla gravida
\end{enumerate}

\clearpage

%------------------------------------------------

\subsection{Verbatim}

How to include a theorem in this presentation:
\begin{verbatim}
\mybox{0.8\textwidth}{
\begin{theorem}[Murphy (1949)]
Anything that can go wrong, will go wrong.
\end{theorem}
}
\end{verbatim}

\clearpage

%------------------------------------------------


\section{Displaying Information}

\clearpage

%------------------------------------------------

\subsection{Table}

\begin{table}[h]
\centering
\begin{tabular}{l l l}
\toprule
\textbf{Treatments} & \textbf{Response 1} & \textbf{Response 2}\\
\midrule
Treatment 1 & 0.0003262 & 0.562 \\
Treatment 2 & 0.0015681 & 0.910 \\
Treatment 3 & 0.0009271 & 0.296 \\
\bottomrule
\end{tabular}
\caption{Table caption}
\end{table}

\clearpage

%------------------------------------------------



%------------------------------------------------

\subsection{Theorem}

The most common definition of \mygreen{Murphy's Law} is as follows.

\mybox{0.8\textwidth}{ % Example of encapsulating text in a colored box
\begin{theorem}[Murphy (1949)]
Anything that can go wrong, will go wrong.
\end{theorem}
}

\begin{proof}
A special case of this theorem is proven in the textbook.
\end{proof}

\begin{remark}
This is a remark.
\end{remark}

\begin{algorithm}
This is an algorithm.
\end{algorithm}

\clearpage

%------------------------------------------------

\section{Citations}

An example of the \texttt{\textbackslash cite} command to cite within the presentation:

This statement requires citation \cite{Smith:2012qr}.

\clearpage

%------------------------------------------------

\thispagestyle{empty} % No slide header and footer

\bibliographystyle{unsrt}
\bibliography{sample}

\clearpage

%------------------------------------------------

\thispagestyle{empty} % No slide header and footer

\begin{tikzpicture}[remember picture,overlay] % Background box
\node [xshift=\paperwidth/2,yshift=\paperheight/2] at (current page.south west)[rectangle,fill,inner sep=0pt,minimum width=\paperwidth,minimum height=\paperheight/3,top color=mygreen,bottom color=mygreen]{}; % Change the height of the box, its colors and position on the page here
\end{tikzpicture}
% Text within the box
\begin{flushright}
\vspace{0.6cm}
\color{white}\sffamily
{\bfseries\LARGE Questions?\par} % Request for questions text
\vfill
\end{flushright}

%----------------------------------------------------------------------------------------

\end{document}
