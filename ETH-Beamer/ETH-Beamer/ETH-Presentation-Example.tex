%  Example for the ETH Beamer template
%  Copyright 2014 by
%  Dr. Antonios Garas,
%  Chair of Systems Design, ETH Zurich
%  Weinbergstrasse 56/58 CH-8092 Zurich
%
\documentclass[
%aspectratio=169,
first,
%handout,
%compress,
%Helv,
ETH1,
navigation
]{ETHbeamerclass}

% Options for beamer:
%
% compress: navigation bar becomes smaller
% t       : place contents of frames on top (alternative: b,c)
% handout : handoutversion
% notes   : show notes
% notes=onlyslideswithnotes
%
\setbeamertemplate{note page}{\ \\[.3cm]
\textbf{\color{colorSG}Notes:}\\%[0.1cm]
{\footnotesize %\tiny
\insertnote}}

\setbeameroption{hide notes}
%\setbeameroption{show notes}

\setbeamertemplate{navigation symbols}{} % suppresses all navigation symbols:
% \setbeamertemplate{navigation symbols}[horizontal] % Organizes the navigation symbols horizontally.
% \setbeamertemplate{navigation symbols}[vertical] % Organizes the navigation symbols vertically.
% \setbeamertemplate{navigation symbols}[only frame symbol] % Shows only the navigational symbol for navigating frames.

\usepackage{amsmath, amssymb}
\usepackage{pifont}
\usepackage{multimedia}

%--------------------------------------------------------------------------------------------------------
%--- Some custom definitions...

\definecolor{cbf}{RGB}{199,100,95}%{163,15,0}
\newcommand{\cbf}{\color{cbf}}

\newcommand{\mean}[1]{\left\langle #1 \right\rangle}
\newcommand{\abs}[1]{\left| #1 \right|}

%--------------------------------------------------------------------------------------------------------
%--- Inputs for the title page...

\title{Distributed Monte Carlo Information\\
Fusion and Distributed Particle Filtering
}
\newcommand{\shorttitle}{Shorter title}

\author{wang junjie}
\date{\today}
\institute{ HIT --- School of Software}

\newcommand{\collaborators}{}
\newcommand{\event}{}
\newcommand{\place}{HIT}

\newcommand{\Prob}{\mathrm{Pr}}
\newcommand{\sign}{\mathrm{sign}}
\newcommand{\R}{\mathbb{R}}
\newcommand{\N}{\mathbb{N}}
\newcommand{\nbin}[1]{\mathrm{nb}_\mathrm{in}(#1)}
\newcommand{\nbout}[1]{\mathrm{nb}_\mathrm{out}(#1)}
\newcommand{\one}{\mathbb{\bf 1}}

\newcommand{\sref}[1]{SLIDE \ref{#1}}

%--------------------------------------------------------------------------------------------------------
%--- Here begins the presentation...

\begin{document}

\frame{
\maketitle
}
\note{}


\section{About}

\frame{

  \frametitle{Contents}
    \tableofcontents[currentsection]
}
\note{}

\frame{
The International Federation of Automatic Control


Cape Town, South Africa. August 24-29, 2014


Australian National University and NICTA, Canberra Research Lab

}


\section{Abstract}

\frame{

  \frametitle{Contents}
    \tableofcontents[currentsection]
}
\note{}
\frame{
   we present a MC solution to the distributed data fusion problem and apply it to distributed particle filter.
}

\section{Introduction}
\frame{

  \frametitle{Contents}
    \tableofcontents[currentsection]
}
\note{}
\frame{
   Our approach to this problem is consensus-based and involves local exchange of posterior densities.The posteriors are exchanged via sets of unweighted particles.
   
   \begin{block}{attractive aspects}
   
    \begin{itemize}
    \item  distributed
    \item  robust
    \item  has a guaranteed speed of convergence
    \item  the resulting posterior distribution is guaranteed to be consistent as long al the dependence between the agents is understood
     
    \end{itemize}
  \end{block}

}

\frame{
    
Hlinka, O., Hlawatsch, F., and Djuri��c, P.M. (2013).\\
Distributed particle filtering in agent networks: A survey,
classification, and comparison. IEEE Signal Processing
Magazine, 30(1), 61�C68.
    
}

\section{Distributed MC data fusion}
\frame{

  \frametitle{Contents}
    \tableofcontents[currentsection]
}
\note{}

\frame{




}

\frame{

  \frametitle{This is the title of the first frame}

  In this frame we use a block:
  \begin{block}{Title of the block}
  Contents of the block...\\
  In one or more lines
    \begin{itemize}
    \item It can also include environments...
     \begin{itemize}
      \item It can also include environments...
     \end{itemize}
    \end{itemize}
  \end{block}

  \begin{block}{}
  Contents of the block...\\
  In one or more lines
    \begin{enumerate}
     \item It can also include environments...
    \end{enumerate}
  \end{block}


}
\note{}

\frame{

  \frametitle{This is the title of the first frame}

  In one or more lines
    \begin{itemize}
    \item It can also include environments...
     \begin{itemize}
      \item It can also include environments...
     \end{itemize}
%    \end{itemize}

%    \begin{itemize}
    \item It can also include environments...
     \begin{itemize}
      \item It can also include environments...
     \end{itemize}
    \end{itemize}

    \begin{enumerate}
     \item It can also include environments...
    \end{enumerate}


}
\note{}

\section{This is section two}

\frame{

  \frametitle{Let's use a long frame title that will be
    displayed in two (or more) lines... This is not a good practice, but it may happen :-)}

  \begin{columns}
   \column{0.45\textwidth}
   Now lets split the presentation using columns..

  \column{0.45\textwidth}
    In this frame we use a block:
    \begin{block}{Title of the block}
    Contents of the block...\\
    In one or more lines
    \begin{itemize}
    \item It can also include environments...
     \begin{itemize}
      \item It can also include environments...
      \end{itemize}
     \end{itemize}
    \end{block}
  \end{columns}
}
\note{}

\frame{

  \frametitle{Contents}
    \tableofcontents[currentsection]
}
\note{}


\frame{

  \frametitle{Summary..}


  }
\note{}


\end{document}
